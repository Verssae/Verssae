%!TEX TS-program = xelatex
%!TEX encoding = UTF-8 Unicode
% Awesome CV LaTeX Template for CV/Resume
%
% This template has been downloaded from:
% https://github.com/posquit0/Awesome-CV
%
% Author:
% Claud D. Park <posquit0.bj@gmail.com>
% http://www.posquit0.com
%
% Template license:
% CC BY-SA 4.0 (https://creativecommons.org/licenses/by-sa/4.0/)
%


%-------------------------------------------------------------------------------
% CONFIGURATIONS
%-------------------------------------------------------------------------------
% A4 paper size by default, use 'letterpaper' for US letter
\documentclass[11pt, a4paper]{awesome-cv}
\usepackage{tcolorbox}

\definecolor{mycolor}{HTML}{ff4b33}  % #ff4b33 색상 정의

\newtcbox{\badge}[1][mycolor]{
  on line, 
  arc=1.5pt,
  colback=#1!80!black,
  colframe=#1!80!black,
  fontupper=\color{white},
  boxrule=1pt, 
  boxsep=0pt,
  left=5pt,
  right=5pt,
  top=1pt,
  bottom=1pt,
  height=12pt,   % 고정 높이 설정
  baseline=3pt   % 기준선을 맞춰줌
}
\newcommand{\githubCode}{\faGithub\space Code}
\newcommand{\preprint}{\faFilePdf[regular]\space Prerpint}
\newcommand{\paper}{\faFilePdf\space Paper}

% \usepackage{xeCJK}
% \setCJKmainfont{Noto Sans KR}
% Configure page margins with geometry
\geometry{left=1.4cm, top=.8cm, right=1.4cm, bottom=1.8cm, footskip=.5cm}

% Color for highlights
% Awesome Colors: awesome-emerald, awesome-skyblue, awesome-red, awesome-pink, awesome-orange
%                 awesome-nephritis, awesome-concrete, awesome-darknight
% \colorlet{awesome}{awesome-emerald}
% Uncomment if you would like to specify your own color
\definecolor{awesome}{HTML}{ff4b33}

% Colors for text
% Uncomment if you would like to specify your own color
% \definecolor{darktext}{HTML}{414141}
% \definecolor{text}{HTML}{333333}
% \definecolor{graytext}{HTML}{5D5D5D}
% \definecolor{lighttext}{HTML}{999999}
% \definecolor{sectiondivider}{HTML}{5D5D5D}

% Set false if you don't want to highlight section with awesome color
\setbool{acvSectionColorHighlight}{true}

% If you would like to change the social information separator from a pipe (|) to something else
\renewcommand{\acvHeaderSocialSep}{\quad\textbar\quad}


%-------------------------------------------------------------------------------
%	PERSONAL INFORMATION
%	Comment any of the lines below if they are not required
%-------------------------------------------------------------------------------
% Available options: circle|rectangle,edge/noedge,left/right
\photo[noedge]{./photo.jpg}

\name{Hansae Ju}{(주한새)}
\position{Software Engineer{\enskip\cdotp\enskip}Machine Learning Researcher}
\address{\faExclamationCircle \space 현재 박사 수료 상태이며, 전문연구요원(현역) 신규 편입 구직 중 입니다!}
% \address{235, World Cup buk-ro, Mapo-gu, Seoul, 03936, Republic of Korea}

\mobile{(+82) 10-5357-2714}
\email{sparky@hanyang.ac.kr}
\dateofbirth{July 15th, 1998}
% \homepage{www.posquit0.com}
\github{Verssae}
\linkedin{juhansae}
% \gitlab{gitlab-id}
% \stackoverflow{SO-id}{SO-name}
% \twitter{@twit}
% \skype{skype-id}
% \reddit{reddit-id}
% \medium{medium-id}
% \kaggle{kaggle-id}
% \hackerrank{hackerrank-id}
% \googlescholar{googlescholar-id}{name-to-display}
%% \firstname and \lastname will be used
% \googlescholar{googlescholar-id}{}
% \extrainfo{extra information}

% \quote{[전문연구요원] Job Application for Software Engineer / Data Scientist / R \& D Engineer}


%-------------------------------------------------------------------------------
\begin{document}

% Print the header with above personal information
% Give optional argument to change alignment(C: center, L: left, R: right)
\makecvheader[R]

% Print the footer with 3 arguments(<left>, <center>, <right>)
% Leave any of these blank if they are not needed
\makecvfooter
  {Updated: \today}
  {Hansae Ju~~·~~~Curriculum Vitae}
  {\thepage}


%-------------------------------------------------------------------------------
%	CV/RESUME CONTENT
%	Each section is imported separately, open each file in turn to modify content
%-------------------------------------------------------------------------------

\begin{cvletter}
    \cvsection{Summary}
    
    한양대학교 인공지능융합학과 박사 수료생입니다. Software Engineering Lab에서 데이터 분석과 자동화 기술을 소프트웨어 엔지니어링에 적용하는 연구를 수행하고 있습니다. 최근 참여한 논문이 소프트웨어 공학 분야 최상위 국제 컨퍼런스인 ASE’24에 선정되었습니다. 또한 웹, 모바일, CLI, 브라우저 확장 프로그램, 게임 등의 소프트웨어를 자동화, 시각화, 모델 인퍼런스와 같은 다양한 요구에 맞춰 개발할 수 있습니다.

    \cvsection{Research Interest}
    
    \entrytitlestyle{Data-driven and Automated Software Engineering}

    소프트웨어 개발 과정에서 생성되는 복잡한 데이터를 분석해 품질과 신뢰성을 향상시키고, 동시에 개발, 유지보수, 테스팅 프로세스를 자동화하여 효율성과 정확성을 높이는 기술과 도구를 연구합니다.
    \vspace{2.0mm}
    \begin{cvitems}
        \item \textbf{Research Areas}: Neurophysiological Software Engineering, Empirical Software Engineering, Mining Software Repositories
        \item \textbf{Technologies}: Static Analysis, Statistical Analysis, Large Language Models, Explainable AI, Reinforcement Learning
        \item \textbf{Applications}: Software Defect Prediction, Software Log Parsing, Software Testing
    \end{cvitems}
    \vspace{2.0mm}
\end{cvletter}

%-------------------------------------------------------------------------------
%	SECTION TITLE
%-------------------------------------------------------------------------------
\cvsection{Education}


%-------------------------------------------------------------------------------
%	CONTENT
%-------------------------------------------------------------------------------
\begin{cventries}

%---------------------------------------------------------
  \cventry
    {PH.D. Candidate in Artificial Intelligence (인공지능융합학과)} % Degree
    {Hanyang University} % Institution
    {Ansan, S.Korea} % Location
    {Mar. 2021 - Feb. 2024} % Date(s)
    {
      \begin{cvitems} % Description(s) bullet points
        \item {Adviced by Prof. Scott Uk-Jin Lee}
        \item {GA: 4.42 / 4.5}
      \end{cvitems}
    }

    \cventry
    {B.S. in Software} % Degree
    {Hanyang University ERICA Campus} % Institution
    {Ansan, S.Korea} % Location
    {Mar. 2017 - Feb. 2021} % Date(s)
    {
      \begin{cvitems} % Description(s) bullet points
        \item {GA: 4.37 / 4.5 (\textbf{Summa Cum Laude})} 
      \end{cvitems}
    }

%---------------------------------------------------------
\end{cventries}

%-------------------------------------------------------------------------------
%	SECTION TITLE
%-------------------------------------------------------------------------------
\cvsection{Experience}


%-------------------------------------------------------------------------------
%	CONTENT
%-------------------------------------------------------------------------------
\begin{cventries}

%---------------------------------------------------------
  \cventry
    {Ph.D. Student} % Job title
    {SELab (Software Engineering Laboratory @한양대학교)} % Organization
    {Ansan, S.Korea} % Location
    {Mar. 2021 - Present} % Date(s)
    {
      \begin{cvitems} % Description(s) of tasks/responsibilities
        \item \textbf{NeuroJIT}: 커밋 레벨의 결함 예측 모델에 코드 이해 과정의 어려움을 반영하는 새로운 접근법. 개발자들의 신경생리학적 반응과 상관관계가 있는 커밋 복잡도를 기반으로 한 피처를 개발 및 분석하기 위한 코드 구현 및 실험 수행
        \item \textbf{Prompt Engineering for Log Parsing}: 로그 파싱 성능 향상을 위해 LLM을 적용하고, 효율성 개선을 위해 프롬프트 엔지니어링 고도화, RAG 파이프라인 구축 등의 연구를 수행
        \item \textbf{Reinforcement Learning for Software Testing}: 복잡한 3D 공간에서 탐색적 테스팅 자동화를 위해 강화학습 알고리즘 연구 수행. Unity를 사용하여 게임 환경 개발, 호기심 기반 보상 알고리즘 및 모방 학습 알고리즘 구현
      \end{cvitems}
    }

%---------------------------------------------------------
% \cventry
% {Undergraduate Intern} % Job title
% {SELab (Software Engineering Laboratory @HYU)} % Organization
% {Ansan, S.Korea} % Location
% {Jul. 2020 - Feb. 2021} % Date(s)
% {
%   \begin{cvitems} % Description(s) of tasks/responsibilities
%     \item {SW중심대학 산학협력프로젝트 | 모바일 게임 성능 최적화 및 효과적인 유지보수 기법 도입 |  \faBuilding[regular] \textit{StudioAttic}}
%   \end{cvitems}
% }
%---------------------------------------------------------

\cventry
{Intern} % Job title
{(주) 솔루게이트} % Organization
{Seoul, S.Korea} % Location
{Jul. 2019 - Aug. 2019} % Date(s)
{
  \begin{cvitems} % Description(s) of tasks/responsibilities
    \item {국세청 음성 데이터 감성 인식을 위한 전처리/전사 작업, Text-to-Speech 딥러닝 모델 학습 및 테스트, 프로토타입 개발 등을 수행. 이 때 한국어 데이터로 Tacotron2 음성합성 모델 학습 시 불편했던 모델 인퍼런스 경험을 개선하고자 Flask 기반의 웹 인터페이스 (\gh{https://github.com/Verssae/flask-tacotron2-tts-web-app}{flask-tacotron2-tts-web-app})를 개발하기도 함}
  \end{cvitems}
}


%---------------------------------------------------------
\end{cventries}


% \cvsection{Extra Experience}

% \begin{cventries}
% \cventry
% {Teaching Assistant} % Job title
% {Hanyang University ERICA Campus} % Organization
% {Ansan, S.Korea} % Location
% {Mar. 2021 - Feb. 2024} % Date(s)
% {
%   \begin{cvitems} % Description(s) of tasks/responsibilities
%     \item {CSE4006 - Software Engineering (2021-1, 2022-1, 2023-1)}
%     \item {CSE2024 - Software Development Practice (2021-2, 2022-2, 2023-2)}
%   \end{cvitems}
% }
% \end{cventries}
%-------------------------------------------------------------------------------
%	SECTION TITLE
%-------------------------------------------------------------------------------
\cvsection{Skills}


%-------------------------------------------------------------------------------
%	CONTENT
%-------------------------------------------------------------------------------
\begin{cvskills}
  
  %---------------------------------------------------------
    \cvskill
      {Programming} % Category
      { Python, JavaScript, Java, C\#} % Skills
%---------------------------------------------------------
  \cvskill
    {DevOps \& PM} % Category
    {Git, Docker, JIRA, Figma} % Skills

%---------------------------------------------------------
  \cvskill
    {ML/DL} % Category
    {scikit-learn, PyTorch, Matplotlib, Pandas } % Skills

  \cvskill
    {Others} % Category
    {Unity, React, React-Native, LaTex} % Skills

%---------------------------------------------------------
\end{cvskills}


%-------------------------------------------------------------------------------
%	SECTION TITLE
%-------------------------------------------------------------------------------
\cvsection{International Conferences}

%-------------------------------------------------------------------------------
%	CONTENT
%-------------------------------------------------------------------------------
\begin{cventries}

%---------------------------------------------------------
\pbentry
  {\faStar[regular] NeuroJIT: Improving Just-In-Time Defect Prediction Using Neurophysiological and Empirical Perceptions of Modern Developers }  %\space \badge{cognitive complexity} \badge{just-in-time defect prediction} \badge{NeuroSE} \badge{machine learning} 
  {Gichan Lee, \textbf{Hansae Ju} and Scott Uk-Jin Lee}
  {\textbf{ASE'24}: The 39th IEEE/ACM International Conference on Automated Software Engineering}
  {\href{https://doi.org/10.1145/3691620.3695056}{\paper}}
  {\href{https://github.com/Verssae/NeuroJIT}{\githubCode}}
  {\textbf{[I6]}}

 \pbentry
  {\faStar[regular] Reliable Online Log Parsing Using Large Language Models with Retrieval-Augmented Generation } %\space \badge{software log analysis } \badge{large language models}
  {\textbf{Hansae Ju}}
  {\textbf{ISSRE'24}: The 35th IEEE International Symposium on Software Reliability Engineering - Doctoral Symposium}
  {\href{https://doi.org/10.1109/ISSREW63542.2024.00055}{\paper}}
  {\href{https://github.com/Verssae/RAGLogParser}{\githubCode}}
  {\textbf{[I5]}}


  \pbentry
  {Enhancing Log Abstraction with Semantic Variable Naming via Large Language Models } %\space \badge{software log analysis } \badge{large language models}
  {\textbf{Hansae Ju}, Joonwoo Lee, Gichan Lee and Scott Uk-Jin Lee}
  {ICEE-CCA'23: International Conference on Electrical Engineering \& Computing Convergence and Applications 2023}
  {\href{https://github.com/Verssae/Verssae/blob/3f89da13976aaa5cbd1e3bdce61926b392d6e797/assets/papers/ICEE-CCA\%2023/Enhancing\%20Log\%20Abstraction\%20with\%20Semantic\%20Variable\%20Naming\%20via\%20Large\%20Language\%20Models.pdf}{\paper}}
  {\href{https://github.com/Verssae/LogParsingGPT}{\githubCode}}
  {[I4]}

  \pbentry
  {Proposal of Efficiency Metric for White-Box Deep Learning Testing } %\space \badge{deep learning testing}
  {Joonwoo Lee, \textbf{Hansae Ju}, Gichan Lee and Scott Uk-Jin Lee}
  {ICEE-CCA'23: International Conference on Electrical Engineering \& Computing Convergence and Applications 2023}
  {\href{https://github.com/Verssae/Verssae/blob/3f89da13976aaa5cbd1e3bdce61926b392d6e797/assets/papers/ICEE-CCA\%2023/Proposal\%20of\%20Efficiency\%20Metric\%20for\%20White-Box\%20Deep\%20Learning\%20Testing.pdf}{\paper}}
  {}
  {[I3]}

  \pbentry
  {Leveraging Prompt Engineering on Large Language Model for Semantic Log Parsing} % \space \badge{software log analysis} \badge{large language models}
  {\textbf{Hansae Ju} and Scott Uk-Jin Lee}
  {ICISCA'23: The 8th International Conference on Information, System and Convergence Applications}
  {\href{https://github.com/Verssae/Verssae/blob/3f89da13976aaa5cbd1e3bdce61926b392d6e797/assets/papers/ICISCA\%2023/Leveraging\%20Prompt\%20Engineering\%20on\%20Large\%20Language\%20Model\%20for\%20Semantic\%20Log\%20Parsing.pdf}{\paper}}
  {\href{https://github.com/Verssae/LogParsingGPT}{\githubCode}}
  {[I2]}

  \pbentry
  {Augmenting Exploratory Testing Agents for 3D Software via Imitation Learning } %\space \badge{software testing} \badge{reinforcement learning}
  {\textbf{Hansae Ju} and Scott Uk-Jin Lee}
  {ICEIC'22: The 21th International Conference on Electronics, Information, and Communication}
  {\href{https://github.com/Verssae/Verssae/blob/3f89da13976aaa5cbd1e3bdce61926b392d6e797/assets/papers/ICEIC\%2022/Augmenting\%20Exploratory\%20Testing\%20Agents\%20for\%203D\%20Software\%20via\%20Imitation\%20Learning.pdf}{\paper}}
  {\href{https://github.com/Verssae/testing-ml-agents}{\githubCode}}
  {[I1]}
%---------------------------------------

%---------------------------------------------------------
\end{cventries}

\cvsection{Domestic Conferences}
\begin{cventries}
  \pbentry
  {개발자 관점의 GitHub 이슈 관행에 대한 경험적 연구 } %\space \badge{empirical study} \badge{issue practices}
  {\textbf{주한새}, Scott Uk-Jin Lee}
  {JCCI'23: 제 31회 통신정보 합동학술대회}
  {\href{https://github.com/Verssae/Verssae/blob/3f89da13976aaa5cbd1e3bdce61926b392d6e797/assets/papers/JCCI\%2023/\%EA\%B0\%9C\%EB\%B0\%9C\%EC\%9E\%90\%20\%EA\%B4\%80\%EC\%A0\%90\%EC\%9D\%98\%20GitHub\%20\%EC\%9D\%B4\%EC\%8A\%88\%20\%EA\%B4\%80\%ED\%96\%89\%EC\%97\%90\%20\%EB\%8C\%80\%ED\%95\%9C\%20\%EA\%B2\%BD\%ED\%97\%98\%EC\%A0\%81\%20\%EC\%97\%B0\%EA\%B5\%AC.pdf}{\paper}}
  {\href{https://github.com/Verssae/GIRT_analysis}{\githubCode}}
  {[D6]}

  \pbentry
  {신재생에너지 발전 연계형 노지형 농장을 위한 저비용 스마트 영농 시스템 } %\space \badge{smart farm} \badge{IoT}
  {\textbf{주한새}, Scott Uk-Jin Lee}
  {KSC'22: 한국소프트웨어종합학술대회}
  {\href{https://www.dbpia.co.kr/Journal/articleDetail?nodeId=NODE11224052}{\paper}}
  {}
  {[D5]}

  \pbentry
  {추상적 텍스트 요약 기반의 메소드 이름 제안 모델 } %\space \badge{method naming} \badge{natural language processing}
  {\textbf{주한새}, Scott Uk-Jin Lee}
  {KSCI'22: 한국컴퓨터정보학회 하계학술대회}
  {\href{https://www.dbpia.co.kr/Journal/articleDetail?nodeId=NODE11140330}{\paper}}
  {}
  {[D4]}

  \pbentry
  {심층강화학습을 활용한 게임엔진 기반 3D 소프트웨어의 탐색적 테스트 자동화 } %\space \badge{software testing} \badge{reinforcement learning}
  {\textbf{주한새}, Scott Uk-Jin Lee}
  {KSC'21: 한국소프트웨어종합학술대회}
  {\href{https://www.dbpia.co.kr/Journal/articleDetail?nodeId=NODE11035659}{\paper}}
  {\href{https://github.com/Verssae/testing-ml-agents}{\githubCode}}
  {[D3]}

  \pbentry
  {소프트웨어 스멜 검출 도구의 낮은 사용률 개선을 위한 연구 } %\space \badge{code smell} \badge{software quality}
  {\textbf{주한새}, Scott Uk-Jin Lee}
  {JCCI'21: 제 31회 통신정보 합동학술대회}
  {\href{https://github.com/Verssae/Verssae/blob/3f89da13976aaa5cbd1e3bdce61926b392d6e797/assets/papers/JCCI\%2021/\%EC\%86\%8C\%ED\%94\%84\%ED\%8A\%B8\%EC\%9B\%A8\%EC\%96\%B4\%20\%EC\%8A\%A4\%EB\%A9\%9C\%20\%EA\%B2\%80\%EC\%B6\%9C\%20\%EB\%8F\%84\%EA\%B5\%AC\%EC\%9D\%98\%20\%EB\%82\%AE\%EC\%9D\%80\%20\%EC\%82\%AC\%EC\%9A\%A9\%EB\%A5\%A0\%20\%EA\%B0\%9C\%EC\%84\%A0\%EC\%9D\%84\%20\%EC\%9C\%84\%ED\%95\%9C\%20\%EC\%97\%B0\%EA\%B5\%AC.pdf}{\paper}}
  {}
  {[D2]}

  \pbentry
  {코드 스멜 탐지 기법 간 효과적 성능 비교 플랫폼 } %\space \badge{code smell} \badge{software quality}
  {\textbf{주한새}, 이기찬, Scott Uk-Jin Lee}
  {KSC'20: 한국소프트웨어종합학술대회}
  {\href{https://www.dbpia.co.kr/Journal/articleDetail?nodeId=NODE10529587}{\paper}}
  {}
  {[D1]}

\end{cventries}
%-------------------------------------------------------------------------------
%	SECTION TITLE
%-------------------------------------------------------------------------------



\cvsection{Software Projects}

\begin{cventries}
  \cventry
  {Personal Project} % Job title
  {Gitflow Visualizer} % Organization
  {\href{https://github.com/Verssae/gitflow-visualizer}{\githubCode}} % Location
  {Aug. 2023 - Sep. 2024} % Date(s)
  {
    \begin{cvitems} % Description(s) of tasks/responsibilities
      \item {GitHub Activity를 기반으로 한 Gitflow 시각화 CLI 프로그램}
      \item {GitHub Workflow 실습 평가 시 팀 별 workflow를 한눈에 파악하기 위한 목적으로 개발}
    \end{cvitems}
  }

  \cventry
  {\faUniversity \space HYU | AIC6025 (EXPLAINABLE ARTIFICIAL INTELLIGENCE) | \faBuilding[regular] NGL Transportation} % Job title
  {DOOCR: Delivery Order OCR} % Organization
  {\href{https://github.com/Verssae/doocr}{\githubCode}} % Location
  {Sep. 2023 - Dec. 2023} % Date(s)
  {
    \begin{cvitems} % Description(s) of tasks/responsibilities
      \item {표 형태를 기반으로 한 물류 주문서 파일을 인식하여, 주문서의 내용을 추출하는 OCR 프로그램}
      \item {Attention Rollout 기반 ViT 모델의 텍스트 인식 결과에 대한 설명가능성 시각화}
    \end{cvitems}
  }

  \cventry
  {\faUniversity \space HYU | CSE9115 (SPECIAL TOPICS IN HEALTHCARE INFORMATION SOFTWARE) | \faHospital[regular] 창원경상국립대학교 병원 } 
  {고령 친화적 시청각 UI를 활용한 효율적인 정형외과 예진 애플리케이션} % Organization
  {\href{https://www.figma.com/design/DY4l2tZ6kkIiZmN0NMk8Lo/Medical?node-id=0-1&t=JGXwtRfDt9z0oH3N-1}{\faFigma\space Figma} \space \href{https://github.com/Verssae/cse9115-medical-app}{\githubCode} } % Location
  {Mar. 2022 - Jun. 2022} % Date(s)
  {
    \begin{cvitems} % Description(s) of tasks/responsibilities
      \item {고령자 친화적 정형외과 예진 앱 개발을 목표로 Figma를 이용한 UI/UX 디자인 및 요구사항 명세}
      \item {React-Native 기반의 모바일 애플리케이션 개발 및 Flask 기반의 예진표 작성 벡엔드 구축}
    \end{cvitems}
  }

  \cventry
  {Personal Project} % Job title
  {Auto Radio Button Checker} % Organization
  {\href{https://chromewebstore.google.com/detail/auto-radio-button-checker/phkflnpejpgehjgficbbikeclfcageic?hl=ko}{\faChrome\space Store} \space \href{https://github.com/Verssae/AutoRadioButtonChecker}{\githubCode} } % Location
  {Sep. 2022} % Date(s)
  {
    \begin{cvitems} % Description(s) of tasks/responsibilities
      \item {웹 페이지의 라디오 버튼 스냅샷을 저장하고 방문 시 자동으로 체크할 수 있는 Chrome 브라우저 확장 프로그램}
      \item {총 다운로드 3,400 건 이상 및 주간 사용자 액티브 수 약 600명}
    \end{cvitems}
  }

  \cventry
  {Personal Project} % Job title
  {날다, 라미} % Organization
  {\href{https://youtu.be/dygD1wzavr8?si=dvC0UiG-BkSIu1Si}{\faYoutube\space Video} \href{https://cheddar-taxicab-6ed.notion.site/199f666eca084723b1e48f366d71fce1}{\faStickyNote\space Docs} \href{https://github.com/Verssae/Rami?tab=readme-ov-file}{\githubCode} } % Location
  {Sep. 2019 ~ Nov. 2019} % Date(s)
  {
    \begin{cvitems} % Description(s) of tasks/responsibilities
      \item {Unity 엔진으로 개발한 2D 캐주얼 액션 게임}
      \item {"2020-2 또래 튜터링 - 프로그래밍언어"에서 튜터로 활동하며 C\# 및 Unity 게임 개발 교육}
    \end{cvitems}
  }

  \cventry
  {Personal Project} % Job title
  {Flask-Tacotron2-TTS-Web-App} % Organization
  {\href{https://github.com/Verssae/flask-tacotron2-tts-web-app}{\githubCode} } % Location
  {Aug. 2019 - May. 2021} % Date(s)
  {
    \begin{cvitems} % Description(s) of tasks/responsibilities
      \item {간편한 Tacotron2 TTS 모델 실행을 위한 Flask 기반의 웹 애플리케이션}
      \item {해외 개발자들과 함께 이슈를 해결한 경험이 있습니다  (GitHub \faStar\space \textbf{28})}
    \end{cvitems}
  }
\end{cventries}
%-------------------------------------------------------------------------------
%	CONTENT
%-------------------------------------------------------------------------------
\cvsection{Research \& Industrial Projects}
\begin{cventries}

%---------------------------------------------------------
% \cventry
%   {인공지능융합혁신인재양성(한양대학교 ERICA) 중점 산학과제} % Job title
%   {공공데이터를 활용한 AI 기반 다품종 작물 재배 시뮬레이션 시스템} % Organization
%   {\faBuilding[regular] (주)블루에너지} % Location
%   {Jan. 2024 - Present} % Date(s)
%   {
%     \begin{cvitems} % Description(s) of tasks/responsibilities
%       \item {컨테이너 스마트팜 구축을 위한 AI 기반 다품종 작물 재배 시뮬레이션 시스템 개발}
%     \end{cvitems}
%   }

%---------------------------------------------------------
\cventry
  {국방기술진흥연구소 [특화연구센터]} % Job title
  {미래 기술 적응형 통합 수중 감시} % Organization
  {\faFlask\space 방위사업청} % Location
  {Nov. 2023 - Present} % Date(s)
  {
    \begin{cvitems} % Description(s) of tasks/responsibilities
      \item {통합 수중 감시 알고리즘 검증 및 다중 센서 모의를 위한 및 Unity 기반의 시각화 시뮬레이터 개발 중}
    \end{cvitems}
  }

%---------------------------------------------------------
\cventry
  {한국연구재단 이공분야기초연구사업 중견연구} % Job title
  {결함 유발 코드 예측의 근거를 설명하는 딥러닝 기반 디버깅 워크벤치 개발} % Organization
  {\faFlask\space 과학기술정보통신부} % Location
  {Mar. 2023 - Present} % Date(s)
  {
    \begin{cvitems} % Description(s) of tasks/responsibilities
      \item { XAI 및 LLM기반의 결함 유발 코드 예측 모델 개발 중}
    \end{cvitems}
  }

%---------------------------------------------------------
\cventry
  {인공지능융합혁신인재양성(한양대학교 ERICA) 중점 산학과제 | \faBuilding[regular] (주)블루에너지} % Job title
  {블루베리 영농을 위한 인공지능 기반 저비용 스마트팜 전환 솔루션} % Organization
  {\faFlask\space 과학기술정보통신부} % Location
  {Jul. 2022 - Dec. 2023} % Date(s)
  {
    \begin{cvitems} % Description(s) of tasks/responsibilities
      \item {라즈베리 파이 및 IoT 센서 기반의 작물 모니터링 체계 구축}
      \item {RS-485{\enskip\cdotp\enskip}Wi-Fi 컨버터를 활용한 태양광 발전량 모니터링 프로그램 개발}
    \end{cvitems}
  }

%---------------------------------------------------------

\cventry
  {인공지능융합연구센터 (한양대학교 ERICA) 전략과제 | \faHospital[regular] 고려대학교 안산병원} % Job title
  {인공지능에 기반한 외래환자 대기 시간 예측} % Organization
  {\faFlask\space 과학기술정보통신부} % Location
  {Jul. 2021 - Dec. 2022} % Date(s)
  {
    \begin{cvitems} % Description(s) of tasks/responsibilities
      \item {고려대학교 안산병원 외래환자 대기 시간 예측을 위한 머신러닝 기반 시계열 예측 모델 개발}
    \end{cvitems}
  }
%---------------------------------------------------------
% \cventry
%   {한국연구재단 이공분야기초연구사업 기본연구} % Job title
%   {코드 스멜을 활용한 딥러닝 기반의 소스 코드 변경 취약성 관리 플랫폼} % Organization
%   {\faFlask\space 과학기술정보통신부} % Location
%   {Jan. 2021 - May. 2021} % Date(s)
%   {
%     \begin{cvitems} % Description(s) of tasks/responsibilities
%       \item {코드 스멜 검출을 통한 딥러닝 기반의 소스 코드 변경 취약성 관리 플랫폼 개발}
%     \end{cvitems}
%   }

% %---------------------------------------------------------
% \cventry
%   {SW중심대학 산학협력프로젝트} % Job title
%   {모바일 게임 성능 최적화 및 효과적인 유지보수 기법 도입} % Organization
%   {\faBuilding[regular] StudioAttic} % Location
%   {Jul. 2020 - Dec. 2020} % Date(s)
%   {
%     \begin{cvitems} % Description(s) of tasks/responsibilities
%       \item {모바일 게임 성능 최적화 및 효과적인 유지보수 기법 도입}
%     \end{cvitems}
%   }
\end{cventries}

%-------------------------------------------------------------------------------
%	SECTION TITLE
%-------------------------------------------------------------------------------
\cvsection{Certificates}


%-------------------------------------------------------------------------------
%	CONTENT
%-------------------------------------------------------------------------------
\begin{cvhonors}

  \cvhonor
    {OPIc} % Name
    {IM3} % Issuer
    {2C8544820458} % Credential ID
    {Sep. 2024} % Date(s)

%---------------------------------------------------------
\end{cvhonors}



%-------------------------------------------------------------------------------
\end{document}
