%-------------------------------------------------------------------------------
%	SECTION TITLE
%-------------------------------------------------------------------------------



% \cvsection{GitHub Projects}

% \begin{cventries}
%   \cventry
%   {Personal Project} % Job title
%   {Gitflow Visualizer} % Organization
%   {\href{https://github.com/Verssae/gitflow-visualizer}{\githubCode}} % Location
%   {Aug. 2023 - Sep. 2024} % Date(s)
%   {
%     \begin{cvitems} % Description(s) of tasks/responsibilities
%       \item {GitHub Activity를 기반으로 한 Gitflow 시각화 CLI 프로그램}
%       \item {GitHub Workflow 실습 평가 시 팀 별 workflow를 한눈에 파악하기 위한 목적으로 개발}
%     \end{cvitems}
%   }

%   \cventry
%   {\faUniversity \space HYU | AIC6025 (EXPLAINABLE ARTIFICIAL INTELLIGENCE) | \faBuilding[regular] NGL Transportation} % Job title
%   {DOOCR: Delivery Order OCR} % Organization
%   {\href{https://github.com/Verssae/doocr}{\githubCode}} % Location
%   {Sep. 2023 - Dec. 2023} % Date(s)
%   {
%     \begin{cvitems} % Description(s) of tasks/responsibilities
%       \item {표 형태를 기반으로 한 물류 주문서 파일을 인식하여, 주문서의 내용을 추출하는 OCR 프로그램}
%       \item {Attention Rollout 기반 ViT 모델의 텍스트 인식 결과에 대한 설명가능성 시각화}
%     \end{cvitems}
%   }

%   \cventry
%   {\faUniversity \space HYU | CSE9115 (SPECIAL TOPICS IN HEALTHCARE INFORMATION SOFTWARE) | \faHospital[regular] 창원경상국립대학교 병원 } 
%   {고령 친화적 시청각 UI를 활용한 효율적인 정형외과 예진 애플리케이션} % Organization
%   {\href{https://www.figma.com/design/DY4l2tZ6kkIiZmN0NMk8Lo/Medical?node-id=0-1&t=JGXwtRfDt9z0oH3N-1}{\faFigma\space Figma} \space \href{https://github.com/Verssae/cse9115-medical-app}{\githubCode} } % Location
%   {Mar. 2022 - Jun. 2022} % Date(s)
%   {
%     \begin{cvitems} % Description(s) of tasks/responsibilities
%       \item {고령자 친화적 정형외과 예진 앱 개발을 목표로 Figma를 이용한 UI/UX 디자인 및 요구사항 명세}
%       \item {React-Native 기반의 모바일 애플리케이션 개발 및 Flask 기반의 예진표 작성 벡엔드 구축}
%     \end{cvitems}
%   }

%   \cventry
%   {Personal Project} % Job title
%   {Auto Radio Button Checker} % Organization
%   {\href{https://chromewebstore.google.com/detail/auto-radio-button-checker/phkflnpejpgehjgficbbikeclfcageic?hl=ko}{\faChrome\space Store} \space \href{https://github.com/Verssae/AutoRadioButtonChecker}{\githubCode} } % Location
%   {Sep. 2022} % Date(s)
%   {
%     \begin{cvitems} % Description(s) of tasks/responsibilities
%       \item {웹 페이지의 라디오 버튼 스냅샷을 저장하고 방문 시 자동으로 체크할 수 있는 Chrome 브라우저 확장 프로그램}
%       \item {총 다운로드 3,400 건 이상 및 주간 사용자 액티브 수 약 600명}
%     \end{cvitems}
%   }

%   \cventry
%   {Personal Project} % Job title
%   {날다, 라미} % Organization
%   {\href{https://youtu.be/dygD1wzavr8?si=dvC0UiG-BkSIu1Si}{\faYoutube\space Video} \href{https://cheddar-taxicab-6ed.notion.site/199f666eca084723b1e48f366d71fce1}{\faStickyNote\space Docs} \href{https://github.com/Verssae/Rami?tab=readme-ov-file}{\githubCode} } % Location
%   {Sep. 2019 ~ Nov. 2019} % Date(s)
%   {
%     \begin{cvitems} % Description(s) of tasks/responsibilities
%       \item {Unity 엔진으로 개발한 2D 캐주얼 액션 게임}
%       \item {"2020-2 또래 튜터링 - 프로그래밍언어"에서 튜터로 활동하며 C\# 및 Unity 게임 개발 교육}
%     \end{cvitems}
%   }

%   \cventry
%   {Personal Project} % Job title
%   {Flask-Tacotron2-TTS-Web-App} % Organization
%   {\href{https://github.com/Verssae/flask-tacotron2-tts-web-app}{\githubCode} } % Location
%   {Aug. 2019 - May. 2021} % Date(s)
%   {
%     \begin{cvitems} % Description(s) of tasks/responsibilities
%       \item {간편한 Tacotron2 TTS 모델 실행을 위한 Flask 기반의 웹 애플리케이션}
%       \item {해외 개발자들과 함께 이슈를 해결한 경험이 있습니다  (GitHub \faStar\space \textbf{28})}
%     \end{cvitems}
%   }
% \end{cventries}
%-------------------------------------------------------------------------------
%	CONTENT
%-------------------------------------------------------------------------------
\cvsection{Research \& Industrial Projects}
\begin{cventries}

%---------------------------------------------------------
% \cventry
%   {인공지능융합혁신인재양성(한양대학교 ERICA) 중점 산학과제} % Job title
%   {공공데이터를 활용한 AI 기반 다품종 작물 재배 시뮬레이션 시스템} % Organization
%   {\faBuilding[regular] (주)블루에너지} % Location
%   {Jan. 2024 - Present} % Date(s)
%   {
%     \begin{cvitems} % Description(s) of tasks/responsibilities
%       \item {컨테이너 스마트팜 구축을 위한 AI 기반 다품종 작물 재배 시뮬레이션 시스템 개발}
%     \end{cvitems}
%   }

%---------------------------------------------------------

  \cventry
  {SW중심대학 (한양대학교 ERICA) | \faBuilding[regular] (주)마이데이터랩} % Job title
  {인공지능 기술 기반 취업 지원 서비스} % Organization
  {\faFlask\space 과학기술정보통신부} % Location
  {Jul. 2024 - Present} % Date(s)
  {
    \begin{cvitems} % Description(s) of tasks/responsibilities
      \item {마이데이터랩의 상용 GPT 기술 기반 취업 지원 서비스(나임: NAIM) 고도화를 위해 취업 정보 맞춤 프롬프트 엔지니어링 및 GPT 파인 튜닝 수행}
    \end{cvitems}
  }

\cventry
  {국방기술진흥연구소 [특화연구센터]} % Job title
  {미래 기술 적응형 통합 수중 감시} % Organization
  {\faFlask\space 방위사업청} % Location
  {Nov. 2023 - Present} % Date(s)
  {
    \begin{cvitems} % Description(s) of tasks/responsibilities
      \item {통합 수중 감시 알고리즘 검증 및 다중 센서 모의를 위한 및 Unity 기반의 시각화 시뮬레이터 개발 중}
    \end{cvitems}
  }

%---------------------------------------------------------
\cventry
  {한국연구재단 이공분야기초연구사업 중견연구} % Job title
  {결함 유발 코드 예측의 근거를 설명하는 딥러닝 기반 디버깅 워크벤치 개발} % Organization
  {\faFlask\space 과학기술정보통신부} % Location
  {Mar. 2023 - Present} % Date(s)
  {
    \begin{cvitems} % Description(s) of tasks/responsibilities
      \item {프롬프트 엔지니어링을 활용하여 결함 유발 코드 예측 및 설명 생성이 가능한 LLM 모델 개발 중}
    \end{cvitems}
  }

%---------------------------------------------------------
\cventry
  {인공지능융합혁신인재양성(한양대학교 ERICA) 중점 산학과제 | \faBuilding[regular] (주)블루에너지} % Job title
  {블루베리 영농을 위한 인공지능 기반 저비용 스마트팜 전환 솔루션} % Organization
  {\faFlask\space 과학기술정보통신부} % Location
  {Jul. 2022 - Dec. 2023} % Date(s)
  {
    \begin{cvitems} % Description(s) of tasks/responsibilities
      \item {IoT 센서 기반의 작물 모니터링 체계 및 RS-485{\enskip\cdotp\enskip}Wi-Fi 컨버터를 활용한 태양광 발전량 모니터링 프로그램 개발}
    \end{cvitems}
  }

%---------------------------------------------------------

\cventry
  {인공지능융합연구센터 (한양대학교 ERICA) 전략과제 | \faHospital[regular] 고려대학교 안산병원} % Job title
  {인공지능에 기반한 외래환자 대기 시간 예측} % Organization
  {\faFlask\space 과학기술정보통신부} % Location
  {Jul. 2021 - Dec. 2022} % Date(s)
  {
    \begin{cvitems} % Description(s) of tasks/responsibilities
      \item {비뇨기과 외래환자 데이터를 기반으로 대기 시간 예측을 위한 시계열 예측 ML 모델 개발. 주로 데이터 분석 및 피쳐 엔지니어링을 수행}
    \end{cvitems}
  }
%---------------------------------------------------------
% \cventry
%   {한국연구재단 이공분야기초연구사업 기본연구} % Job title
%   {코드 스멜을 활용한 딥러닝 기반의 소스 코드 변경 취약성 관리 플랫폼} % Organization
%   {\faFlask\space 과학기술정보통신부} % Location
%   {Jan. 2021 - May. 2021} % Date(s)
%   {
%     \begin{cvitems} % Description(s) of tasks/responsibilities
%       \item {코드 스멜 검출을 통한 딥러닝 기반의 소스 코드 변경 취약성 관리 플랫폼 개발}
%     \end{cvitems}
%   }

% %---------------------------------------------------------
% \cventry
%   {SW중심대학 산학협력프로젝트} % Job title
%   {모바일 게임 성능 최적화 및 효과적인 유지보수 기법 도입} % Organization
%   {\faBuilding[regular] StudioAttic} % Location
%   {Jul. 2020 - Dec. 2020} % Date(s)
%   {
%     \begin{cvitems} % Description(s) of tasks/responsibilities
%       \item {모바일 게임 성능 최적화 및 효과적인 유지보수 기법 도입}
%     \end{cvitems}
%   }
\end{cventries}
